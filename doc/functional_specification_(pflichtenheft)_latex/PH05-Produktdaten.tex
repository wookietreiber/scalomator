\section{Produktdaten}

%\comment{Was speichert das Produkt (langfristig) aus Benutzersicht?}

\begin{description}
  \item[/D010/]
    \textit{Automatendefinition:} Die Struktur eines endlichen Automaten, bestehend aus Zust�nden und �bergangen, wird in einer XML-Datei gespeichert, die der folgenden Definition gen�gt.
    \lstset{language=XML,morecomment=[s]{!--}{--}}
\begin{lstlisting}%[caption=XSD]
<?xml version="1.0" encoding="UTF-8"?>
<xs:schema xmlns:xs="http://www.w3.org/2001/XMLSchema"
    elementFormDefault="qualified">

  <xs:element name="finitestatemachine">
    <xs:complexType>
      <xs:sequence>
        <xs:element ref="initialstate"/>
        <xs:element ref="finalstates"/>
        <xs:element ref="transitions"/>
      </xs:sequence>
      <xs:attribute name="name" type="xs:normalizedString"/>
    </xs:complexType>
  </xs:element>

  <xs:element name="initialstate" type="xs:normalizedString"/>

  <xs:element name="finalstates">
    <xs:complexType>
      <xs:sequence>
        <xs:element maxOccurs="unbounded" ref="finalstate"/>
      </xs:sequence>
    </xs:complexType>
  </xs:element>

  <xs:element name="finalstate" type="xs:normalizedString"/>

  <xs:element name="transitions">
    <xs:complexType>
      <xs:sequence>
        <xs:element maxOccurs="unbounded" ref="transition"/>
      </xs:sequence>
    </xs:complexType>
  </xs:element>

  <xs:element name="transition">
    <xs:complexType>
      <xs:sequence>
        <xs:element ref="start"/>
        <xs:element ref="input"/>
        <xs:element ref="ends"/>
      </xs:sequence>
    </xs:complexType>
  </xs:element>

  <xs:element name="start" type="xs:normalizedString"/>

  <xs:element name="input" type="xs:normalizedString"/>

  <xs:element name="ends">
    <xs:complexType>
      <xs:sequence>
        <xs:element maxOccurs="unbounded" ref="end"/>
      </xs:sequence>
    </xs:complexType>
  </xs:element>

  <xs:element name="end" type="xs:normalizedString"/>

</xs:schema>
\end{lstlisting}
\end{description}

