\section{Globale Testszenarien und Testf�lle}

Jede Produktfunktion \textit{/F????/} wird anhand von konkreten Testf�llen \textit{/T????/} getestet.\\
Die dabei verwendeten Namen werden rein zuf�llig gew�hlt.

\begin{description}
\item[/T0010/]
	\textit{Zustand erstellen:} Es werden Zust�nde mit den Namen "`0"', "`1"', "`2"', "`4"' und "`5"' erstellt.
\item[/T0020/]
	\textit{Zustand l�schen:} Der Zustand "`4"' wird gel�scht.
\item[/T0030/]
	\textit{Zustand umbenennen:} Der Zustand "`5"' wird in "`3"' umbenannt.
\item[/T0040/]
	\textit{Zustand verschieben:} Der Zustand "`0"' wird verschoben.
\item[/T0100/]
	\textit{�bergang erstellen:} Ein �bergang vom Zustand "`0"' zum Zustand "`1"' wird erstellt mit dem Eingabezeichen "`a"'.
	Ein weiterer Zustands�bergang von "`1"' nach "`2"' mit dem Eingabezeichen "`b"' wird erstellt.
\item[/T0110/]
	\textit{�bergang l�schen:} Der �bergang von "`1"' nach "`2"' wird gel�scht.
\item[/T0120/]
	\textit{�bergang bearbeiten:} Beim �bergang von "`0"' nach "`1"' wird das Eingabezeichen auf "`b"' ge�ndert.
\item[/T0200/]
	\textit{Zustand zu Startzustand erkl�ren:} Der Zustand "`0"' wird zum Startzustand erkl�rt.
\item[/T0200/]
	\textit{Zustand zu Endzustand erkl�ren:} Die Zust�nde "`1"' und "`2"' werden zu Endzust�nden erkl�rt.
\item[/T0200/]
	\textit{Startzustand oder Endzustand zu normalen Zustand erkl�ren:} Der Zustand "`2"' wird wieder zum normalen Zustand erkl�rt.

\item[/T0300/]
	\textit{Automatendefinition speichern:} Die Automatendefinition wird gespeichert.
\item[/T0320/]
	\textit{Neue Automatendefinition erstellen:} Der aktuell bearbeitete Graph wird verworfen.
\item[/T0310/]
	\textit{Automatendefinition laden:} Der zuvor gespeicherte Automat wird geladen.

\item[/T0400/]
	\textit{Simulation ausf�hren:} Starten der Simulation mit dem geladenen Automaten mit dem Eingabewort "`b"'.
\item[/T0410/]
	\textit{Simulationsergebnis anzeigen:} Anzeige des Ergebnisses. Der Automat sollte das Eingabewort akzeptieren.

\item[/T1000/]
	\textit{Ausgabe einer Beschreibung der Programmparameter:} Aufruf der Kommandozeilenapplikation mit dem Parameter "`-help"'.
\item[/T1010/]
	\textit{Ausf�hrung der Simulation:} Start der Konsolenapplikation, so dass diese den zuvor gespeicherten Automaten l�dt und f�r das Eingabewort "`a"'.
	Das Ergebnis sollte lauten, dass der Automat das Eingabewort nicht akzeptiert.
\end{description}
