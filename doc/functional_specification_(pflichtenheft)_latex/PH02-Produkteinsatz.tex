\section{Produkteinsatz}

\comment{Welche Anwendungsbereiche (Zweck), Zielgruppen (Wer mit welchen Qualifikationen), Betriebsbedingungen (Betriebszeit, Aufsicht)?}

\subsection{Anwendungsbereiche}

Einzelpersonen verwenden dieses Programm zur Simulation endlicher Automaten.

%Einzelpersonen verwenden diesen Dienst zum Spielen der oben angegebenen Brettspiele mit anderen Personen der Spielgemeinschaft.
%Diese Plattform soll dem Einzelnen eine Kommunikation mit Gleichgesinnten erm�glichen,
%um so ihre Fertigkeiten im Spiel verbessern zu k�nnen.

\subsection{Zielgruppen}

Studenten, die im Bereich der theoretischen Informatik ausgebildet werden, sollen ihre Kenntnisse im Fachgebiet der Automatentheorie mittels dieses Programms vertiefen und festigen.
Zuvor per Hand ermittelte Ergebnisse k�nnen von den Studenten �berpr�ft werden.

%Personengruppen, die kurz zur Ablenkung z.B. in der Mittagspause, gerne an Fernspielen teilhaben,
%in dem sie sich Gedanken �ber bevorstehende Spielz�ge machen k�nnen.\\
%Diese Plattform ist f�r Einzelpersonen gedacht, die in ihrer knapp bemessenen Freizeit Schwierigkeiten haben,
%ihrem Hobby z.B. Schach nachzugehen oder Gegner zu finden.\\
%\\
%Es werden Basiskenntnisse in Internetnutzung vorausgesetzt. Ebenso die Spielregeln des jeweiligen Spieltyps sollten vor der
%Nutzung bekannt sein.\\
%\\
Soweit keine weiteren Sprachen integriert sind, muss der Benutzer die Verkehrssprache \textit{Englisch} zumindest verstehen.

\subsection{Betriebsbedingungen}

Dieses System soll sich bez�glich der Betriebsbedingungen nicht wesentlich von anderen Anwendungen unterscheiden.

\begin{itemize}
  \item Betriebsdauer: wenn ben�tigt
  \item wartungsfrei
\end{itemize}
