\section{Zielbestimmungen}
\textbf{Scalomator} ist ein Computerprogramm, welches die Ausf�hrung endlicher Automaten auf eine Eingabe simuliert. Hierbei sollen sowohl deterministische endliche Automaten (deterministic final automaton, DFA) als auch nichtdeterministische endliche Automaten (nondeterministic finite automaton, NFA) simuliert werden k�nnen.
\\\\
Das Programm soll der Unterst�tzung der Ausbildung von Studenten der Studienrichtung Informatik an der Staatlichen Studienakademie Leipzig im Fachgebiet der theoretischen Informatik dienen.

\subsection{Theoretischer Hintergrund}

Ein \textbf{Automat} ist in der Informatik das Modell eines digitalen, zeitdiskreten Rechners. Ob es m�glich oder sinnvoll ist, eine solche Maschine tats�chlich zu bauen, ist dabei zun�chst unerheblich. Die Vereinfachung der F�higkeiten erlaubt es, das Verhalten eines Automaten leichter zu verstehen und zu vergleichen.
\\\\
Das grunds�tzliche Verhalten eines Automaten ist immer gleich: Dem Automaten wird von au�en eine Eingabe als Folge von Zeichen vorgelegt. Der Automat befindet sich in einem Zustand. Jedes Mal, wenn ein Eingabezeichen eintrifft, kann sich abh�ngig vom Eingabezeichen und dem gegenw�rtigen Zustand ein neuer Zustand, der Folgezustand, einstellen (Zustands�bergang oder Transition). Man kann die Menge der m�glichen Zustands�berg�nge, die das Verhalten des Automaten definiert, als das Programm des Automaten verstehen.
\\\\
Wenn der Folgezustand durch den gegenw�rtigen Zustand und das Eingabezeichen immer eindeutig gegeben ist, dann spricht man von einem \textbf{deterministischen Automaten}. Allgemein aber kann man auch einen Spielraum (Freiheitsgrade) f�r die Zustands�berg�nge zulassen. Der Automat darf dann auf dasselbe Paar von Zustand und Eingabezeichen unter mehreren m�glichen Kandidaten einen Folgezustand willk�rlich w�hlen. Dann spricht man von einem \textbf{nichtdeterministischen Automaten}. Der Nichtdeterminismus ist dann willkommen, wenn man das Verhalten der Umgebung modellieren m�chte, das man nicht v�llig genau kennt (don't know), oder wenn man M�glichkeiten f�r verschiedene Implementierungen offenlassen m�chte (don't care).
\\\\
Ein Automat hei�t endlich, wenn die Menge der Zust�nde, die er annehmen kann, endlich ist. Ein \textbf{endlicher Automat} ist ein Spezialfall aus der Menge der Automaten. Ein Zustand speichert die Information �ber die Vergangenheit, d.h. er reflektiert die �nderungen der Eingabe seit dem Systemstart bis zum aktuellen Zeitpunkt. Ein Zustands�bergang zeigt eine �nderung des Zustandes des endlichen Automaten und wird durch logische Bedingungen beschrieben, die erf�llt sein m�ssen, um den �bergang zu erm�glichen. Eine Aktion ist die Ausgabe des endlichen Automaten, die in einer bestimmten Situation erfolgt. 
\\\\
Ein endlicher Automat wird definiert durch ein 5-Tupel $(\Sigma, S, s_0, \delta, E)$, wobei:
\begin{itemize}
\item[$\Sigma$] ist das Eingabealphabet (eine endliche nicht leere Menge von Symbolen),
\item[$S$] ist eine endliche nicht leere Menge von Zust�nden,
\item[$s_0$] ist der Anfangszustand, $s_0\in S$,
\item[$\delta$] ist die Zustands�bergangsfunktion: $\delta: S \times \Sigma \rightarrow S$,
\item[$E$] ist die Menge von Endzust�nden, $E \subseteq S$
\end{itemize}

\subsection{Musskriterien}

\begin{itemize}
  \item Berechnungsprogramm
    \begin{itemize}
      \item Entgegennahme von Automatendefinition.
      \item Entgegennahme von Eingabewort.
      \item Berechnung der Akzeptanz des Automaten f�r dieses Eingabewort.
      \item Ausgabe der Akzeptanz (Simulationsergebnis).
    \end{itemize}
  \item Grafische Benutzeroberfl�che
    \begin{itemize}
      \item Erstellen von Automatendefinitionen durch grafische Oberfl�che.
      \item Speicherung von erstellten Automatendefinitionen auf Speichermedien.
      \item Laden von bereits erstellten Automatendefinitionen von Speichermedien.
      \item Start des Berechnungsprogramms und Anzeige des Simulationsergebnisses.
    \end{itemize}
\end{itemize}

\subsection{Wunschkriterien}

\begin{itemize}
  \item Eine Ausgabe des Simulationsablaufs.
\end{itemize}

\subsection{Abgrenzungskriterien}

\begin{itemize}
  \item Keine Simulation von Kellerautomaten oder Turingmaschinen.
\end{itemize}
